\documentclass{article}

% if you need to pass options to natbib, use, e.g.:
%     \PassOptionsToPackage{numbers, compress}{natbib}
% before loading neurips_2018

% ready for submission
% \usepackage{neurips_2018}

% to compile a preprint version, e.g., for submission to arXiv, add add the
% [preprint] option:
%     \usepackage[preprint]{neurips_2018}

% to compile a camera-ready version, add the [final] option, e.g.:
     \usepackage[final]{nips}

% to avoid loading the natbib package, add option nonatbib:
%     \usepackage[nonatbib]{neurips_2018}

\usepackage[utf8]{inputenc} % allow utf-8 input
\usepackage[T1]{fontenc}    % use 8-bit T1 fonts
\usepackage{hyperref}       % hyperlinks
\usepackage{url}            % simple URL typesetting
\usepackage{booktabs}       % professional-quality tables
\usepackage{amsfonts}       % blackboard math symbols
\usepackage{nicefrac}       % compact symbols for 1/2, etc.
\usepackage{microtype}      % microtypography

\title{Movie Rating Prediction Using Machine Learning}

% The \author macro works with any number of authors. There are two commands
% used to separate the names and addresses of multiple authors: \And and \AND.
%
% Using \And between authors leaves it to LaTeX to determine where to break the
% lines. Using \AND forces a line break at that point. So, if LaTeX puts 3 of 4
% authors names on the first line, and the last on the second line, try using
% \AND instead of \And before the third author name.

\author{%
  Grigor ~Keropyan\thanks{https://github.com/grigor97} \\
  Department of Mathematics and Mechanics\\
  Yerevan State University\\
  \texttt{goqorkeropyan@gmail.com} \\
}

\begin{document}
% \nipsfinalcopy is no longer used

\maketitle

\begin{abstract}
  Movie rating prediction based on information available prior to theatrical release is important in order to understand how successful will be movie. This paper describes various Machine Learning methods to predict movie rating. 
\end{abstract}

\section{Introduction}

Movie rating prediction is becoming a popular problem and various methods have been suggested. In this paper we will introduce Machine Learning (ML) methods to predict movie rating based on the data available prior to the theatrical release. We used IMDB open-source data such as posters and run-time of movies. \\

Although, it seems impossible to predict the movie rating based on the information available before theatrical release, we will see that ML methods predict them quite accurately. One of the explanation could be that people may like movies if their preferred actor is in the crew or films are produced by their favorite producers.\\

One of the goals of this work is provide a tool which can help producers to promote their films to be successful. Another one is that this work provide a good recommendation for people predicting their IMDB score. 

\section{Related Work}
\label{rel_work}

This work is based on the Stanford students paper [1] where they predict IMDB score of movies. In [1] authors used ML based methods to predict IMDB score. Another research which uses Bayesian approach [2]. This paper is based on [1] and uses new ML algorithms such as lightgbm to reach the better accuracy than [1]. Our preproccessing is almost the same as [1] and we have about 20K examples. We have shuffled the dataset and splited into training and test test correspondingly 80 and 20 precents. In the algorithm we have done K-Folds cross validation and after tuning the hyperparamters we have achieved better accuracy than [1]. 

\section{Dataset and Features}
\label{headings}

We have used open source data, "Movie Genre from its Poster Dataset" [3] and "The Movie Dataset" [4]. The dataset is preprocessed like [1], taking only movies which original language is english and which produced after 1980. The features are divided into 3 groups: text (synopses), images (posters) and others (runtime, genre, director, actors). From poster we have extracted some features: such as number of peoples in the poster, mean and standard deviation of blue, green, red, hue, saturation, brightness as in [5]. For summaries we have used count vectorizers and only left the word which appeared more than 20 times. Directors are have been made one hot and actors count vectorizers where we have only left top three actors in each movie. The example of result is shown in Table 1. 

\begin{table}
  \caption{Dataset by Groups}
  \label{dataset-table}
  \centering
  \begin{tabular}{llll}
    \toprule
    Data & Type & Dimension & Example  \\
    \midrule
    Actors & Categorical  & 1671 &  Tom Hanks   \\
    poster & Numerical & 13 & Number of faces = 0 \\
    Director & Categorical & 491 & Howard Deutch \\
    Runtime & Numerical & 1 & 121 (minutes) \\
    Genre & Categorical & 23 & Documentation \\
    Synopses & Categorical & 3712 & Once \\
    \bottomrule
  \end{tabular}
\end{table}
After filtering the english movies and released after 1980 it remained 18850 movies. Which we have splited in two ways. First: like [1] implementation, when it is separated train, validation and test sets correspondingly 70, 15, 15 percents. Second: which gave a better result it is separated into train and test sets correspondingly 80 and 20 percents. 

\section{Methods}
\label{others}

These instructions apply to everyone.

\subsection{Citations within the text}

The \verb+natbib+ package will be loaded for you by default.  Citations may be
author/year or numeric, as long as you maintain internal consistency.  As to the
format of the references themselves, any style is acceptable as long as it is
used consistently.

The documentation for \verb+natbib+ may be found at
\begin{center}
  \url{http://mirrors.ctan.org/macros/latex/contrib/natbib/natnotes.pdf}
\end{center}
Of note is the command \verb+\citet+, which produces citations appropriate for
use in inline text.  For example,
\begin{verbatim}
   \citet{hasselmo} investigated\dots
\end{verbatim}
produces
\begin{quote}
  Hasselmo, et al.\ (1995) investigated\dots
\end{quote}

If you wish to load the \verb+natbib+ package with options, you may add the
following before loading the \verb+neurips_2018+ package:
\begin{verbatim}
   \PassOptionsToPackage{options}{natbib}
\end{verbatim}

If \verb+natbib+ clashes with another package you load, you can add the optional
argument \verb+nonatbib+ when loading the style file:
\begin{verbatim}
   \usepackage[nonatbib]{neurips_2018}
\end{verbatim}

As submission is double blind, refer to your own published work in the third
person. That is, use ``In the previous work of Jones et al.\ [4],'' not ``In our
previous work [4].'' If you cite your other papers that are not widely available
(e.g., a journal paper under review), use anonymous author names in the
citation, e.g., an author of the form ``A.\ Anonymous.''

\subsection{Footnotes}

Footnotes should be used sparingly.  If you do require a footnote, indicate
footnotes with a number\footnote{Sample of the first footnote.} in the
text. Place the footnotes at the bottom of the page on which they appear.
Precede the footnote with a horizontal rule of 2~inches (12~picas).

Note that footnotes are properly typeset \emph{after} punctuation
marks.\footnote{As in this example.}

\subsection{Figures}

\begin{figure}
  \centering
  \fbox{\rule[-.5cm]{0cm}{4cm} \rule[-.5cm]{4cm}{0cm}}
  \caption{Sample figure caption.}
\end{figure}

All artwork must be neat, clean, and legible. Lines should be dark enough for
purposes of reproduction. The figure number and caption always appear after the
figure. Place one line space before the figure caption and one line space after
the figure. The figure caption should be lower case (except for first word and
proper nouns); figures are numbered consecutively.

You may use color figures.  However, it is best for the figure captions and the
paper body to be legible if the paper is printed in either black/white or in
color.

\subsection{Tables}

All tables must be centered, neat, clean and legible.  The table number and
title always appear before the table.  See Table~\ref{sample-table}.

Place one line space before the table title, one line space after the
table title, and one line space after the table. The table title must
be lower case (except for first word and proper nouns); tables are
numbered consecutively.

Note that publication-quality tables \emph{do not contain vertical rules.} We
strongly suggest the use of the \verb+booktabs+ package, which allows for
typesetting high-quality, professional tables:
\begin{center}
  \url{https://www.ctan.org/pkg/booktabs}
\end{center}
This package was used to typeset Table~\ref{sample-table}.

\begin{table}
  \caption{Sample table title}
  \label{sample-table}
  \centering
  \begin{tabular}{lll}
    \toprule
    \multicolumn{2}{c}{Part}                   \\
    \cmidrule(r){1-2}
    Name     & Description     & Size ($\mu$m) \\
    \midrule
    Dendrite & Input terminal  & $\sim$100     \\
    Axon     & Output terminal & $\sim$10      \\
    Soma     & Cell body       & up to $10^6$  \\
    \bottomrule
  \end{tabular}
\end{table}

\section{Final instructions}

Do not change any aspects of the formatting parameters in the style files.  In
particular, do not modify the width or length of the rectangle the text should
fit into, and do not change font sizes (except perhaps in the
\textbf{References} section; see below). Please note that pages should be
numbered.

\section{Preparing PDF files}

Please prepare submission files with paper size ``US Letter,'' and not, for
example, ``A4.''

Fonts were the main cause of problems in the past years. Your PDF file must only
contain Type 1 or Embedded TrueType fonts. Here are a few instructions to
achieve this.

\begin{itemize}

\item You should directly generate PDF files using \verb+pdflatex+.

\item You can check which fonts a PDF files uses.  In Acrobat Reader, select the
  menu Files$>$Document Properties$>$Fonts and select Show All Fonts. You can
  also use the program \verb+pdffonts+ which comes with \verb+xpdf+ and is
  available out-of-the-box on most Linux machines.

\item The IEEE has recommendations for generating PDF files whose fonts are also
  acceptable for NeurIPS. Please see
  \url{http://www.emfield.org/icuwb2010/downloads/IEEE-PDF-SpecV32.pdf}

\item \verb+xfig+ "patterned" shapes are implemented with bitmap fonts.  Use
  "solid" shapes instead.

\item The \verb+\bbold+ package almost always uses bitmap fonts.  You should use
  the equivalent AMS Fonts:
\begin{verbatim}
   \usepackage{amsfonts}
\end{verbatim}
followed by, e.g., \verb+\mathbb{R}+, \verb+\mathbb{N}+, or \verb+\mathbb{C}+
for $\mathbb{R}$, $\mathbb{N}$ or $\mathbb{C}$.  You can also use the following
workaround for reals, natural and complex:
\begin{verbatim}
   \newcommand{\RR}{I\!\!R} %real numbers
   \newcommand{\Nat}{I\!\!N} %natural numbers
   \newcommand{\CC}{I\!\!\!\!C} %complex numbers
\end{verbatim}
Note that \verb+amsfonts+ is automatically loaded by the \verb+amssymb+ package.

\end{itemize}

If your file contains type 3 fonts or non embedded TrueType fonts, we will ask
you to fix it.

\subsection{Margins in \LaTeX{}}

Most of the margin problems come from figures positioned by hand using
\verb+\special+ or other commands. We suggest using the command
\verb+\includegraphics+ from the \verb+graphicx+ package. Always specify the
figure width as a multiple of the line width as in the example below:
\begin{verbatim}
   \usepackage[pdftex]{graphicx} ...
   \includegraphics[width=0.8\linewidth]{myfile.pdf}
\end{verbatim}
See Section 4.4 in the graphics bundle documentation
(\url{http://mirrors.ctan.org/macros/latex/required/graphics/grfguide.pdf})

A number of width problems arise when \LaTeX{} cannot properly hyphenate a
line. Please give LaTeX hyphenation hints using the \verb+\-+ command when
necessary.

\subsubsection*{Acknowledgments}

Use unnumbered third level headings for the acknowledgments. All acknowledgments
go at the end of the paper. Do not include acknowledgments in the anonymized
submission, only in the final paper.

\section*{References}
\label{reference}

\small

[1] Yichen Yang et al., {\it "Predicting Movie Ratings with Multimodal Data"},
\url{http://cs229.stanford.edu/proj2019aut/data/assignment_308832_raw/26260680.pdf}.

[2] Y. J. Lim and Y. W. Teh, {\it "Variational bayesian approach to movie rating prediction"}  Proceedings of KDD Cup and Workshop, vol. 7, 2007, pp. 15–21.

[3] KaggleInc, {\it “Movie genre from its poster,”} 
\url{https://www.kaggle.com/neha1703/movie-genre-from-its-poster} .

[4] Kaggle, {\it “The movies dataset,” } \url{https://www.kaggle.com/rounakbanik/the-movies-dataset} .

[5] F. B. Moghaddam, M. Elahi, R. Hosseini, C. Trattner, and M. Tkalciˇ c, {\it“Predicting movie popularity and ratings with visual features,” } in 2019 14th International Workshop on Semantic and Social Media Adaptation and Personalization (SMAP). IEEE, 2019, pp. 1–6.

\end{document}